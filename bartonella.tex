\documentclass[12pt]{article}
\usepackage{graphicx}
\title{Case Control study into the association of B. henselae infection with Feline Uveitis.}
\author{Hugh Crockford}
\date{\today}
\begin{document}
	\maketitle

\begin{abstract}
	This study is awesome.
\end{abstract}

	\section{Introduction}
		Uveitis is a delibitating disease that can leave cats blind. 
		Clinical signs of Uveitis are the same regardless of cause and can include aqueous flare, iritis, keratic precipitates, hyphema, and hypopyon \cite{Powell2010}.
		The most common causes of endogenous feline uveitis include \emph{Bartonella sp.} , toxoplasmosis, feline immunodeficiency virus (FIV), lymphosarcoma (LSA) with or without feline leukemia virus (FeLV), feline infectious peritonitis (FIP), cryptococcosis, and idiopathic causes \cite{Powell2001}.
		Identifying the cause of uveitis can be difficult due to the non specific clincal signs, and the lack of adequote diagnostic test \cite{Fontenelle2008}.
		Diagnostic tests include Serology via Immunofluorescent assay, Enzyme Linked Immunosorbant assay,or Western Blot, and identifying the organism by culture or PCR.
		PCR and culture can be performed on both blood and Aqueus humor, in an attempt to establish causation by locating \emph{Bartonella sp.} at the site of inflammation. Unfortunately \emph{Bartonella sp.} levels in blood and Aqueus humor fluctuate during infection and infected cats may still show a negative test, and hence multiple test are indicated\cite{Guptill2010}. Likewise amplyfying DNA from the eye does not nececarily imply causation, and can indicate contamination during sample collection \cite{Powell2010}.
		
		\emph{Bartonella sp.} is a common transient infection ni cats, with most not showing any clinical signs. 
		transmitted by fleas
		On average 20 percent of cats are seropositive to \emph{Bartonella sp.}, although this varies widely with geographic location\cite{Jameson1995a}. Areas with warm, humid environments have higher seroprevalence, and these areas are good environments for fleas, indicating the importance of fleas as a vector.
		The association of \emph{Bartonella sp.} and uveitis was first reported by Lappin et al\cite{Lappin1999}, and subsequent studies have found conflicting results, with Ketring reporting increased serum antibodies to \emph{Bartonella sp.} in cats with uveitis\cite{Ketring2004}, however Fontanelle found cats without uveitis more likely to have \emph{Bartonella sp.} antibodies than cats with uveitis.
		
		\emph{Bartonella sp.} has also been implicated in cases of chronic stomatits, anemia, CNS disorders\cite{Nasir2005}.
		\emph{Bartonella sp.} is an important pathogen from a human health perspective as well, being responsible for Cat Scratch Disease, 
		\emph{Bartonella sp.} also causes ocular complications in 5 to 10 percent of people that become infected through Cat Scratch Disease \cite{Wade2000}.
		The reference population for this study is , and study findings will be generalisable to the ~85 million privately owned cats in the United States \cite{HSUSown}.

		\subsection{Objectives and Study Hypothesis}
			The Objective of this study are as follows: 
				\begin{itemize}
					\item To determine the prevalence of \emph{Bartonella sp.} exposure in cats presenting to commercial and university veterinary hospitals for annual checkup.
					\item To study the relationship between infection with \emph{Bartonella sp.} and feline uveitis.
				\end{itemize}
			The Study Hypothesis is there is a higher incidence of feline uveitis amoung cats that have been exposed to \emph{Bartonella sp.}.
\section{Methods}
	Utilising a commercial database (Banfield corporation, Mars inc
	Cases will be taken from Banfields inhouse opthalmology service records, and the Veterinary Medical Database \cite{UniversityVeterinary}.

	Exposure will be measured with a comercially available Western immunoblot test \cite{febart}. This test is widely used and correlates more closely with the ability to isolate \emph{Bartonella sp.} from cats than does the Immunofluorescent assay or Enzyme Linked Immunosorbant assay \cite{Jr1995}. 
	Western Blot results of 3+ and 4+ will be considered positive for the purposes of this study. 


	Case definition will be uveitis positive as diagnosed by tono.. with a cutoff of .
	No evidence of trauma, cataract, intraocular neoplasia, or corneal ulceration (as detected by flureocin stain) 
	after eliminating all other causes of uveitis and positive Western Blot test results
	Cases will have blood taken and tested for FIV/FeLV as these diseases can also cause uveitis.

	Blood will be taken from these cats ( with the written consent of the owner) and sumitted to UC Davis Laboratory for Serology and BLood culture.
	All sample collection in cases will take place before any therapy is instituted.

	Sample size was requirements were calculated using the prevalence figure observed in part one of this study..

	Controls will be taken from cats visiting banfield hospitals for annual checkups. As a part of their checkup, cats are routinely tested for FIV and FeLV , and Toxoplasmosis according to company policy. This removes the potential confounder of FIV/FeLV which is another cause of uveitis in cats. These cats represent a broud sample of the reference population, that could themselves develop uveitis as a result of \emph{Bartonella sp.} exposure.
	

	\subsection{Statistical Evaluation}
		Age, geographic location, housing status, and \emph{Bartonella sp.} status were entered into ..
		Cats were grouped into 4 groups (<2, 2-5, 5-10, >10) as \emph{Bartonella sp.} infection is more common in younger cats
		flea risk was catagorically assigned high or low, based on a combination of indoor/outdoor status and state of origin (Alaska, Arizona, Colorado, Idaho, Montana, Nevada, New Mexico, Utah, and Wyoming all assigned low risk, all other states high, according to \cite{Jameson1995a})
		Cases and controls were matched for age and geographical location (Each University Veterinary Hospital and the Banfield hospitals in the same state)

	% Controls - two groups - healthy and submitted to hospital?


	Ethical permission was granted by UC Davis School of Veterinary Medicine. appropriate forms. can be found..

\section{Results}
\section{Discussion}
	\subsection{Strengths and Limitations}
		Using cats from the same reference population is better than previous studies that utilised cats from an animal shelter. Cats in a shelter are more likely to have been outdoors and exposed to fleas, and hence be seropositive, reducing any observed difference.
		Using indoor/outdoor status as proxy for flea exposure was chosen to reduce the . 
		This design coul dbe imporved with actual flea exposure history from owners, although there is potential for misclassification based on owners not admitting to fleas.

\newpage
\bibliographystyle{unsrt}
\bibliography{bart.bib}

\end{document}

%%%%%
% need to establish prev of uveitis/bartonella seroprevalence in popn - restrict to just cats for checkup and vaccination, no trauma/sick animals allowed.
% screening test for FIV, FeLV
% compare to feral cats - shelters? higher flea risk.
%% cats admitting to banfield - indoor, looked after, less flea risk.
%% previous prevalence studes low n - 55, 47. levy. ( seroprev and PCR respective.y, indicating active infection.)
% zoonosis - justify w human health? Cat scrathc dz.

%%  need numbers of cats seen per year at banfield.

% sencon objetive - ticks and bartonella?
\emph{Bartonella sp.}
% all cats w uveitis - blood collected in to EDTA tubes, frozen, and culture

% need to establish prev in control control group and then establish sample size based on OR you want - (2 - doubling fo risk)
% interactoins - 
%% make streamlined as possible. but need at least one confounder, not necc any interactions, can say did not find any (kim paper for technique - none stat sig so not included). potential effect modify - wualitative. 
% if include need to show OR with and without interactions.
% table 1 - cats in study,. throw in other factors that wont include - make them the same. e.g. age w exposure outcome and not on causal path. - no sig p value. 
% fig 1 - exposure b \emph{Bartonella sp.} , outcome ev.

% exclusion critera - other dz that cause uveitis.- FIV FeLV. 
%% can presnet hypothetical - e.g. FIV . not an issue in this study but may be.

% can justify N in footnote w equation.

% diagram bartonella causing uveitis, age associated w bartonella and uveitis (double ended arrow.) then show in table.( anythign in table that same do not need to have in diagram.

% reference population - randomy selected from same geograph popn, will be owned by person who takes to vet
%% watch biases when choosing controls

% adress in study limitation - only cats that are looked after - findings only apply to this 
% strength - most cats that would be worried about

% actual diagnosis - dichotomise - cutoff. same tests on cases and controls. be redundant.

% summ up - further needed - take to cohort, clinical trial

% data QA - redundancy, masking, using trained people. maks statistition.

% strength - high data Quality techniques.

% run several logistic regressions 
